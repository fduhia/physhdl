% Copyright (C) 2012, Shawn Tan <shawn.tan@sybreon.com>
% Permission is granted to copy, distribute and/or modify this document
% under the terms of the GNU Free Documentation License, Version 1.3
% or any later version published by the Free Software Foundation;
% with no Invariant Sections, no Front-Cover Texts, and no Back-Cover Texts.
% A copy of the license is included in the section entitled "GNU
% Free Documentation License".

\chapter*{Preface}

I started writing this book as a result of teaching a subject called Digital Systems \& HDL.
Through the years, I have come across a myriad of texts that cover the topic but never quite in the way that I found useful.
I have noticed that while people were generally capable of learning a language and understanding digital systems, they often lacked the necessary ability to link the language to the physical hardware implementation.
% That is what motivated me to write this book.

% What is this book about and how is it different from other texts?
% It will be easier to describe what this book isn't rather than is.

Many people typically approach HDL based design just like they approach software programming but that would be a \emph{disaster}.
While the two may look similar, they are fundamentally different and using one like the other will invariably result in badly designed systems.

To avoid any doubt, this is not a book on hardware description languages (HDL).
There are many texts on the various HDLs available, particularly for popular ones, such as Verilog and VHDL.
Those who are interested to learn the intricacies and subtleties of each language should refer to those established texts.
In Part \ref{PART:HDL} of this book, I merely review a limited sub-set of these languages, for those who may need to learn just the bare essentials of the language to understand the rest of the book.

In a similar vein, this is not a book on digital systems design.
There are many texts on digital systems design, at both the undergraduate and graduate levels.
However, these texts are typically general in nature and do not necessarily focus on HDL based designs.
In Part \ref{PART:SYS} of this book, I will graphically show how each component of a digital system can be designed with HDL using cookbook-styled examples.

More importantly, this is not a book on chip design.
There are other texts that are more detailed on this topic, specifically those provided by various vendors that are specifically tailored to their individual design processes.
In Part \ref{PART:FASM} of this book, I will set down a general design flow that is used for digital systems design using HDL.
This is essential to our understanding of where HDLs feature in the larger scheme of things.


I'm not saying that I know the best way to do things, but this is the way that I do things and it seems to work.
Therefore, 
% I decided to write a book on this subject to share my thoughts, ideas, and experiences with the world.
I hope that it will be of some use to others out there.
